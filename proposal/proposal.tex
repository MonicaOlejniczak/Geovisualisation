% Initialisation
\documentclass[a4paper]{article}

% Enforce UTF8 encoding
\usepackage[utf8]{inputenc}

% Header and footer
\usepackage{fancyhdr}
 
\pagestyle{fancy}
\fancyhead[R]{SENG4800 - Individual project proposal}
\fancyfoot[C]{\thepage}
\renewcommand{\headrulewidth}{0pt}
\renewcommand{\footrulewidth}{0pt}

% Document title and author
\title{3D visualisation of large data in a multi-platform environment}
\author{Monica Olejniczak}

% Paragraphing
\setlength{\parindent}{0em}
\setlength{\parskip}{1em}

\begin{document}

	% Title page
	\makeatletter
	\begin{titlepage}
		\vspace*{\fill}
		\begin{center}
			\Huge\textbf{\textsf{\@title}}
			\rule{\textwidth}{1pt}
			\LARGE\textsc{Final year project} \\
			\Large\textsc{University of Newcastle}
			\rule{\textwidth}{1pt}
			\Large\textbf{\textsf{\@author}}
        \end{center}
		\vspace*{\fill}
	\end{titlepage}
	\makeatother
	
	\section{Project description} {
	
		Data visualisation is the presentation of data in a visual context and is particularly important in data analytics since humans are able to recognise patterns, trends and correlations more easily.
		
		This project aims to develop cross-platform visualisations in three-dimensions and will focus on using large data in a generic way to demonstrate general applicability. Ideally, the visualisations will provide touch-capability so they are able to be viewed on a smart device.
	
	}
	
	\section{Basic deliverables} {

		\begin{itemize}
			\item In-depth topic literature review.
			\item Build the implementation for a single visualisation with a small dataset.
		\end{itemize}
	
	}
	
	\section{Intermediate deliverables} {
	
		\begin{itemize}
			\item Incorporate interactivity such as the ability to rotate, pan and zoom.
			\item Develop a large dataset for the developed visualisation.
			\item Build the implementation for another visualisation with a large dataset.
		\end{itemize}	
	
	}
	
	\section{Advanced deliverables} {
	
		\begin{itemize}	
			\item Port the visualisations to a touch-interface.
			\item Incorporate touch gestures.
			\item Demonstrate general applicability with a different dataset.
			\item Analyse the computational power needed for the visualisations.
			\item Perform a user study to compare what types of visualisations are suited for particular purposes.
		\end{itemize}
	
	}
	
	\section{Approval for \today} {
	
		\vspace{1em}\par
		\textbf{Student:} 
		\vspace{2em}\par
		\textbf{Supervisor:}
		
	}
	
\end{document}
