%!TEX root = ../../report.tex

\begin{lstlisting}[language=GLSL,basicstyle=\scriptsize\ttfamily]
	vec3 basicColor(vec2 uBound, vec2 uColorRange, float uMagnitude, float uSaturation, float uLightness) {
		float hue = clamp(convertRange(uBound, uColorRange, uMagnitude), 0.0, 1.0);
		return hsv2rgb(vec3(hue, uSaturation, uValue));
	}

	vec3 gradientColor(vec2 uBound, vec3 vPosition, vec3 uLowColor, vec3 uMediumColor, vec3 uHighColor) {
		// Calculate the bound and colour stop.
		float bound = abs(uBound.y - uBound.x);
		float colorStop = bound * 0.3;
		// Get the current y position of the vertex.
		float y = vPosition.y;
		// Return the gradient colour.
	    return mix(mix(uLowColor, uMediumColor, smoothstep(uBound.x, colorStop, y)), uHighColor, smoothstep(colorStop, uBound.y, y));
	}

	void main() {
		vec3 color;
		if (uMode == 0) {
			color = basicColor(uBound, uColorRange, uMagnitude, uSaturation, uValue);
		} else {
			color = gradientColor(uBound, vPosition, uLowColor, uMediumColor, uHighColor);
		}
		gl_FragColor = vec4(color, uAlpha);
	}
\end{lstlisting}
