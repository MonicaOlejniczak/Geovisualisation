\begin{abstract}

	\todo{reword at end for current project with results, etc}

	Geovisualisation is the interactive visualisation of geospatial information. Visualising this information is particularly important in data analytics, because it aids data exploration and decision-making processes when combined with human understanding.

	This project will attempt to address how to visualise large datasets in WebGL and will analyse how different datasets can be applied to 3D geovisualisations. These considerations are important in data analytics because the web is becoming a prominent medium for publishing geospatial data. This data can then be analysed through geovisualisations for a variety of application domains. Applying different datasets to geovisualisations will facilitate increased data exploration and decision-making processes. Furthermore, this project will compare the differences in performance and scalability between the visualisations and the datasets applied to them.

	To adhere to these project requirements, two visualisations will be developed with WebGL using earth and social science datasets. These visualisations need to be developed quickly, and thus a rapid application development approach is most suitable for this project. Finally, it is critical that the prototypes are developed to be highly modular, performant, scalable and testable to cater to the success of this project.

\end{abstract}
