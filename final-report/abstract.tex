%!TEX root = report.tex

\begin{abstract}

	The web has become a prominent medium for developing geovisualisations, the visualisation of geospatial information, by promoting the creation of highly interactive and immersive environments. Additionally the visualisation of geospatial data is particularly important in the field of data analytics, as geovisualisations facilitate data exploration and decision-making processes when combined with human understanding. This thesis has proposed a method for developing interactive 3D geovisualisations using HTML5 and WebGL to promote the analysis of geospatial data. In order to evaluate whether the system is an effective tool for data analysis, it is essential to consider the usability and performance of the geovisualisations across large datasets.

	This thesis has introduced two types of geovisualisations that were developed using a rapid application development methodology. Another visualisation was created, based off the first geovisualisation, which was integrated into the final year group project as a teaching analytics component. These visualisations can be applied to various datasets by preprocessing the data into an object-based JSON format. The system also features navigation interactions, information displays, filtering tools and custom environment configurations which attempt to further the analysis of geospatial data.

	The system was evaluated by conducting a user study and performance analysis. The results of the user study found that the system was overall easy to use and very consistent. However, users did not feel confident using the system and many features in the user interface and navigation controls could be improved to provide a more positive user experience. Similarly, the performance analysis revealed that the application startup time and run-time performance declined significantly as the size of the dataset increased.

	It was concluded that this system forms a good basis for a prototype to be used in data analytics. However, the system needs several improvements toward its usability and performance before it can be considered an effective tool for the analysis of geospatial data. Further work is recommended to make improvements in the system developed, which have been considered in the future work of this thesis.

\end{abstract}
