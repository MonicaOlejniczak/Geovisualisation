%!TEX root = ../report.tex

This chapter first summarises the thesis in Section~\ref{sec:summary}, then moves onto the signficance, contributions and limitations of this work in Section~\ref{sec:conclusions}. Lastly the directions for future work in Section~\ref{sec:future_work} is discussed.

\section{Thesis summary} {
\label{sec:summary}

	A general introduction to Geovisualisation

}

\section{Conclusions} {
\label{sec:conclusions}

	Through this work, 

	% \item How can a geovisualisation be visualised in WebGL?
	% \item How does a geovisualisation perform across small and large datasets?
	% \item How can different datasets be applied to 3D geovisualisations?
	% \item Do particular colours have any significant effect on the usability of a geovisualisation?
	% \item Do particular filters aid in the analysis of a geovisualisation?

% 	The work described in this thesis has been concerned with the development of an image segmentation algorithm for linear structures within a stochastic Bayesian framework. A new prior model consisting of a set of random trees was proposed to represent the vascular structure of retinal fundus images. An investigation of the
% Markov chain Monte Carlo simulation technique was presented together with the
% Metropolis-Hastings-Green updates, which were used to implement a set of local
% and global moves that alter the tree model. A number of interesting features
% of the proposed algorithm have been described and the method was shown to
% be effective and robust on vessel segmentation from noisy data. These may be
% summarised as:

	overall usability of the system could be improved, especially by providing a small tutorial when opening the system and help tooltips.	

	can't currently compare dataset with one's from years ahead

}

\section{Future work} {
\label{sec:future_work}

	This thesis has demonstrated the potential of geovisualisations by creating visualisations that can be applied to geospatial datasets. As outlined in the previous section, there are still many limitations with the system that present opportunities for improvement and extending the scope of this thesis.

	One direction for future work would be to load multiple datasets into a visualisation which can be viewed, compared and analysed by the user in the same browser window. Currently, the user would have to load these datasets separately in different instances of the application and compare the visualisations side by side in separate browser windows. These datasets could be compared by creating split screen windows within the application for each visualisation with a different dataset. Otherwise the user could select a dataset, which would load the data and replace the contents of the scene. In this case, the height of the data points could be animated if there were variations between the datasets that did not effect the x, y or z coordinates, such as the year that the data was recorded.

	Additionally, a network analysis should be organised in the future. This analysis would be used to measure the expected application startup time for the user when this system is deployed to a production environment.

	Another area of future work would be to improve the usability of the visualisations. This would involve conducting extensive research into good user interface practices and correcting the known issues with the system discovered from the user study. Other navigation techniques should be investigated to improve the rotation of visualisations and the navigation techniques should be specific to the visualisation. For example, in the population globe the rotation should be performed on the surface instead of the camera. The system should also include an in-built tutorial for first time users, clearer text for information checkboxes and labelled axes for the grid surface. Furthermore, another user study would be performed to evaluate the changes in the system in order to analyse and determine if the usability has improved.

	Finally, the performance of the system could be improved significantly. The first change would be to offload the dataset processing to the Node.js server, which would decrease the load time of the application. Another method for decreasing the startup time would be to develop production scripts and compress texture resources. Moreover, the geometries should be merged and an alternate object picking method should be investigated as a way of improving the performance of the system.

}
