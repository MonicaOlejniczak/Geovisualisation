%!TEX root = ../report.tex

This chapter serves as a foundation for the implementation details outlined in Chapter~\ref{ch:implementation} by establishing the technologies used in this project. The development environment is first outlined, before moving onto software configuration management. Next, the CSS pre-processors and stylesheet languages are considered. Finally, the libraries utilised in the system implementation are noted including node.js, three.js, RequireJS, Backbone, Handlebars and the Material Design Framework.

\section{Development environment} {
\label{sec:development_environment}

	The tools and libraries used for the development environment of this project have been outlined in Table~\ref{tab:development_environment}.

	\begin{table}[H]
\caption[Development environment]{The list of project dependencies for the development environment.}
\label{tab:development_environment}
\begin{tabularx}{\textwidth}{@{}XX@{}}
	\toprule
	\textbf{Dependency} & \textbf{Description} \\
	\midrule
	Git & Version control \\
	BitBucket & Source code host \\
	SourceTree & Git client \\
	IntelliJ IDEA & Integrated development environment \\
	Node.js & Runtime environment \\
	Three.js & WebGL abstraction library \\
	RequireJS & Dependency injection library \\
	\bottomrule
\end{tabularx}
\end{table}

}

\section{Software configuration management} {
\label{sec:software_configuration_management}

	\emph{Git} is a distributed revision control system that has been used throughout the course of this project. Its primary function is to manage changes in the source code, but also to maintain any associated documentation for this project. The key advantages of using this system are as follows:

	\begin{itemize}
		\item Emphasis on speed and scale.
			\begin{itemize}
				\item Can support thousands of contributors.
			\end{itemize}
		\item Provides great flexibility towards workflows.
		\item Low storage requirements.
		\item It is decentralised.
			\begin{itemize}
				\item Promotes offline work because everybody has their own repository.
			\end{itemize}
	\end{itemize}

	It is important to note that a decentralised system is superior to other revision control systems, in that changes can be saved locally and then later be added to the remote repository. On the other hand, a centralised system such as Subversion, requires users to physically copy their changes in code when the network location is out of reach.

	While this project is being implemented by a single developer, it is essential to consider future work and the possibility of having more contributors. By using Git as a means of revision control, this should not become an issue due to its sheer scalability.

}

\section{CSS} {
\label{sec:css}

	Less~\footnote{\bibentry{sellier2009less}} is a CSS pre-processor that extends the CSS language by allowing variables, mixins, functions and other techniques to generate CSS that is more maintainable, themable and extendable. Less was used in this project over Sass due to its simplicity. Sass is more powerful than Less, but requires the installation of Ruby to compile \texttt{scss} files. Less is more intuitive than Sass and incorporates all the necessary features that would be used in this project. Consequently, this makes Less the more feasible and optimal solution for creating CSS and as a result has been used for the system implementation.

}

\section{Libraries} {
\label{sec:libraries}
	
	\subsection{Node.js} {
	\label{sec:nodejs}
	
		Node.js is an open-source, cross-platform runtime environment for network applications. It uses an event-driven architecture and non-blocking I/O model that makes it lightweight and efficient~\footnote{\bibentry{nodejs2009node}}. This platform is ideal for web projects due to its simplicity and quick deployment time, as an application can be created, built and run in a matter of minutes.

	}

	\subsection{Three.js} {
	\label{sec:threejs}

		Three.js~\footnote{\bibentry{cabello2010three}} is a JavaScript library that abstracts WebGL and was used for developing the visualisations. It is superior to other WebGL frameworks because of its strong communnity, well-structured codebase, extensive set of features and examples that can be integrated into any project. Libraries such as SceneJS are not designed for rendering complex scenes, while GLGE and other available libraries are less feature complete. Therefore, this makes Three.js a great candidate for developing this project.

	}

	\subsection{RequireJS} {
	\label{sec:requirejs}
		
		RequireJS is a JavaScript file and module loader~\footnote{\bibentry{chung2011requirejs}} that can be used as a dependency injection library. This library promotes the use of modules, so a JavaScript application can resemble a typical class structure in other languages. It also has an optimiser that can generate a single script for the entire application, which can be used in production mode.

	}

	\subsection{Backbone} {
	\label{sec:backbone}

		Backbone is a lightweight, fast and easy to learn MVC framework, which has been primarily used for the storage of models and collections due to its simplicity and RESTful JSON interface. Other MVC frameworks, such as Ember and AngularJS, are more feature-rich but are considerably less lightweight. While these other frameworks support a greater variety of features, the system design specifications determined that these features would not be used at any point during the project. Backbone has also been incorporated into the group project which has increased familiarity with the technology throughout the development of this system. Moreover, this facilitates integration between the two systems as this process can be accomplished more easily when the same frameworks are used. Therefore, Backbone became a primary candidate for model implementation.

	}

	\subsection{Handlebars} {
	\label{sec:handlebars}

		Handlebars is a JavaScript web templating system that was used for the widget and information displays in this project. It is built on top of Mustache, which is often considered a base for JavaScript templating\footnote{\bibentry{franklin2013template}}. This templating system was preferred to others because it is widely used, has a large community and helper methods can be registered and accessed within a template. Furthermore, this same library was also in the group project, increasing consistency and integration between the systems.

		The DOM structure could have also been generated by dynamically creating HTML on the client side. However, this method is less elegant and harder to configure. The use of HTML templates facilitates a clean, structured syntax which has been obtained by using Handlebars.

	}

	\subsection{Material Design framework} {
	\label{sec:material_design_framework}

		There are several CSS3 Material Design frameworks available such as Material Design for Bootstrap, Material Design Lite and Materialize. Each framework has been examined below.

		\emph{Material Design for Bootstrap} is a theme for Bootstrap 3, which lets you use Google Material Design on top of Bootstrap~\footnote{\bibentry{vrasta2014material}}. It is currently used in the group project, but it is a bloated framework and requires the following:

		\begin{verbatim}
			jquery.min.js 		29.3KB
			bootstrap.min.css 	19.2KB 
			material.min.css 	23.9KB
			material.min.js 	1.6KB
		\end{verbatim}

		This totals \texttt{74KB} gzipped, which is a reasonably large strain on the network for loading a single library on a system that targets good performance.
		
		\emph{Material Design Lite} is a lightweight Material Design framework, that doesn't rely on any JavaScript frameworks~\footnote{\bibentry{google2014materialdesignlite}}. It provides a good variety of components, including a sliding drawer which other Material Design frameworks do not support. However, this framework has a bloated syntax and the drawer could not be configured precisely as intended. It is also extremely difficult to change colour schemes when installing this framework on your own server.

		\emph{Materialize} is a modern, responsive front-end framework based on Material Design~\footnote{\bibentry{materialize2014materializecss}}. This framework provides a more elegant syntax than Material Design for Bootstrap and Material Design Lite, and can have a customised colour scheme. While this framework is ideal in many ways, the components that will be used in this project only consist of sliders and checkboxes. These components are not styled or animated in a way that is quite suitable for this project.

		Materialize is a promising framework from the above options. However, since so few components are being used in this project, it was decided to manually design the components to reflect the Material Design specifications. This method provides the most flexibility as components are fully customised and it is also the most lightweight solution.

	}

}

\section{Pipeline} {
\label{sec:technology_pipeline}

	A high level view of the technology pipeline has been demonstrated in Figure~\ref{fig:technology_pipeline}. The Node.js web server listens to a specific port that can be accessed in the web browser, where the \texttt{index.html} page acts as the main entry point of the system. This entry point immediately loads RequireJS and the Application. The Application defines the routes to each dependency, so they can be loaded by RequireJS when requested and as necessary. The Application uses THREE.js to render the components in a scene on the HTML5 canvas which is displayed on the web page. The Application also makes use of the Handlebars library to add helper methods for Handlebars view templates. These methods are generally used to add formatting and display logic to the view. Moreover, the Application uses Backbone Collections, Models and ViewControllers. Collections and Models are used to store a processed dataset, while the Application appends the HTML generated from a ViewController, which compiles and renders a Handlebars view template.

	%!TEX root = ../../report.tex

\begin{figure}[H]
	\centering
    \includegraphics[width=\textwidth]{images/technologies/pipeline}
    \caption{Technology pipeline.}
    \label{fig:technology_pipeline}
\end{figure}


}
