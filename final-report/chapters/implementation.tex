\todo{Purpose of the chapter
 Structure of the chapter
 Central themes of the chapter}

\section{Development environment} {
\label{sec:development_environment}

	The tools and libraries used for the development environment of this project have been outlined in Table~\ref{tab:development_environment}.

	\begin{table}[H]
\caption[Development environment]{The list of project dependencies for the development environment.}
\label{tab:development_environment}
\begin{tabularx}{\textwidth}{@{}XX@{}}
	\toprule
	\textbf{Dependency} & \textbf{Description} \\
	\midrule
	Git & Version control \\
	BitBucket & Source code host \\
	SourceTree & Git client \\
	IntelliJ IDEA & Integrated development environment \\
	Node.js & Runtime environment \\
	Three.js & WebGL abstraction library \\
	RequireJS & Dependency injection library \\
	\bottomrule
\end{tabularx}
\end{table}

}

\section{Software configuration management} {
\label{sec:software_configuration_management}

	\emph{Git} is a distributed revision control system that has been used throughout the course of this project. Its primary function is to manage changes in the source code, but also to maintain any associated documentation for this project. The key advantages of using this system are as follows:

	\begin{itemize}
		\item Emphasis on speed and scale.
			\begin{itemize}
				\item Can support thousands of contributors.
			\end{itemize}
		\item Provides great flexibility towards workflows.
		\item Low storage requirements.
		\item It is decentralised.
			\begin{itemize}
				\item Promotes offline work because everybody has their own repository.
			\end{itemize}
	\end{itemize}

	It is important to note that a decentralised system is superior to other revision control systems, in that changes can be saved locally and then later be added to the remote repository. On the other hand, a centralised system such as Subversion, requires users to physically copy their changes in code when the network location is out of reach.

	While this project is being implemented by a single developer, it is essential to consider future work and the possibility of having more contributors. By using Git as a means of revision control, this should not become an issue due to its sheer scalability.

}

\section{Libraries} {
\label{sec:libraries}
	
	\subsection{Node.js} {
	\label{sec:nodejs}
	
		Runtime environment

	}

	\subsection{Three.js} {
	\label{sec:threejs}

		Three.js~\footnote{\bibentry{cabello2010three}} is a JavaScript library that abstracts WebGL. It is superior to other WebGL frameworks because of its strong communnity, well-structured codebase, extensive set of features and examples that can be integrated into any project. Libraries such as SceneJS are not designed for rendering complex scenes, while GLGE and other available libraries are less feature complete. Therefore, this makes Three.js a great candidate for developing this project.

	}

	\subsection{RequireJS} {
	\label{sec:requirejs}
		
		Dependency injection library

	}

	\subsection{Handlebars} {
	\label{sec:handlebars}

		Handlebars is a JavaScript web templating system that was used for the information displays in this project. It is built on top of Mustache, which is often considered a base for JavaScript templating\footnote{\bibentry{franklin2013template}}. This templating system was preferred to others because it is widely used, has a large community and helper methods can be registered and accessed within a template. Furthermore, this same library was also in the group project, increasing consistency and integration between the systems.

		The information displays could have also been generated by dynamically creating HTML in the client side. However, this method is less elegant and harder to configure. The use of HTML templates facilitates a clean, structured syntax which has been obtained by using Handlebars.

	}

}

\section{Material Design framework} {
\label{sec:material_design_framework}
	
	\todo{evaluate material design lite, materialize, bootstrap material}

}
	
\section{Navigation} {
\label{sec:navigation}

	\todo{math formulas for spherical coords}

	\subsection{Pan} {
	\label{sec:pan}

	}

	\subsection{Rotate} {
	\label{sec:rotate}
	
	}

	\subsection{Zoom} {
	\label{sec:zoom}
	
	}

}

\subsection{Filtering} {

	\todo{say how filtering was done}

}

\subsection{Configuration} {

	\todo{talk about dat.gui and its ui transition -- updated display}

	\todo{shaders}

}

\section{Data display} {
\label{sec:data_display}

	\todo{write about how the data displays were implemented}

	% The development of these visualisations will involve using the established \href{http://threejs.org/docs/#Reference/Extras.Geometries/BoxGeometry}{BoxGeometry} and NURBS surface that are available in Three.js. The \href{http://threejs.org/examples/webgl_geometry_nurbs.html}{NURBS example}, as displayed in Appendix~\ref{app:nurbs}, proves that it is possible to render a smooth 3D surface. This can be applied to represent a heat map and if there lie difficulties in implementing this, then a simpler representation can be modelled.

}

\section{Information display} {
\label{sec:information_display}



}

\section{Flat surface} {
\label{sec:flat_surface}
	
}

\section{Round surface} {
\label{sec:round_surface}
	
}

\section{Grid} {
\label{sec:grid}
	
}